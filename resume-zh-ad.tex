\documentclass[10pt,a4paper,roman]{moderncv}

% moderncv themes
\moderncvstyle{classic}
\moderncvcolor{black}
%\renewcommand{\familydefault}{\sfdefault}
%\nopagenumbers{}
% character encoding
%\usepackage[utf8]{inputenc}
%\usepackage{CJKutf8}

% adjust the page margins
\usepackage[scale=0.85]{geometry}
\setlength{\hintscolumnwidth}{0.13\textwidth}
\setlength{\makecvtitlenamewidth}{8cm}

\usepackage{fontspec}
\usepackage{xunicode}
\usepackage{xeCJK}
\setmainfont{Times New Roman}
\setsansfont{Times New Roman}
\setmonofont{Courier New}
%\setCJKmainfont{Adobe Kaiti Std}
\setCJKmainfont[BoldFont={Kaiti SC Bold}]{Kaiti SC}
%\setCJKmainfont{Kaiti SC}
\setCJKsansfont{Kaiti SC}
\setCJKmonofont{SimHei}

\usepackage{lastpage}
\usepackage{fancyhdr}
\pagestyle{fancy}
\fancyhf{}

\usepackage{datetime2}
%\newdateformat{UKvardate}{%
%\THEDAY\ \monthname[\THEMONTH] \THEYEAR}
%\UKvardate

\usepackage{manfnt}

\newcommand{\hello}{{\tiny\textdbend}}

% to show numerical labels in the bibliography (default is to show no labels); only useful if you make citations in your resume
%\makeatletter
%\renewcommand*{\bibliographyitemlabel}{\@biblabel{\arabic{enumiv}}}
%\makeatother

% bibliography with mutiple entries
%\usepackage{multibib}
%\newcites{book,misc}{{Books},{Others}}
%----------------------------------------------------------------------------------
%            content
%----------------------------------------------------------------------------------

\AfterPreamble{\hypersetup{
    pdfstartview={XYZ null null 1.30},
    baseurl={http://}
}}

\fancyfoot[L]{\footnotesize\textit{陈晓辉的个人简历}}
\fancyfoot[C]{\footnotesize\thepage/\pageref{LastPage}}
\fancyfoot[R]{\footnotesize\textit{上次更新:\today}}

% personal data
\name{陈晓辉}{}
\title{男, 27}
\address{北京市海淀区科学院南路2号院3号楼搜狐媒体大厦 100190}{}{}
\phone[mobile]{+86~138~1179~8428}
%\phone[fixed]{+86~027~6875~5072} % optional
%\phone[fax]{+86~027~6875~4150} % optional
\email{sdqxcxh@gmail.com}
\homepage{www.cxh.me}
%\extrainfo{}
\social[github]{chenxiaohui}
\social[linkedin]{chenxiaohui}
%\photo[64pt][0.4pt]{picture}
%\quote{{\footnotesize Last updated: \today}}

\begin{document}
\makecvtitle

\section{教育经历}
\cvitem{2010.9--2013.6 }{ 清华大学软件学院,软件工程硕士, 方向:领域语义模型存储}
\cvitem{2006.9--2010.6 }{ 北京航空航大学计算机学院,计算机科学与技术本科}


\section{工作经历}
\cvitem{2014.11--今}{ 搜狐大数据中心推荐系统和精准广告团队,高级研发工程师(后台系统开发),L3.1}
\cvitem{2013.7--2014.10 }{ 支付宝北京基础数据部OceanBase团队,研发工程师(分布式数据库、分布式事务),P5}


\section{外语能力}
\cvitem{\hello}{具备良好的英语听说读写能力,熟练阅读英文文献。大学英语四级(548)、六级(549)均已通过。}


\section{获奖情况}
\cvitem{2010 }{ 北京航空航天大学优秀毕业生}
\cvitem{2010 }{ 北京航空航天大学计算机学院科技实践奖学金}
\cvitem{2009 }{ 北航第十九届“冯如杯”学生学术科技作品竞赛二等}
\cvitem{2008 }{ 北京市奥运会志愿者先进个人}


\section{个人陈述}
\cvitem{\hello}{专业兴趣浓厚,热衷于钻研各种技术,参与了很多项目,实践能力和学习能力均比较强,乐于分享,长期维护技术博客;极客,崇尚高效率快节奏的生活方式,喜欢对现有方法进行改进以提高效率。}
\cvitem{\hello}{受过规范的开发流程、做事风格和代码风格训练,有较强的团队意识和一定的组织领导能力,独立领导完成过中小型项目。扎实肯干,易于相处。}


\section{专业素质}
\cvitem{\hello}{多年c/c++经验,熟悉主要分布式算法,参与设计实现过分布式存储系统,对关系数据库理论、NoSQL有一定了解和经验。}
\cvitem{\hello}{对开源分布式存储系统有浓厚兴趣,了解redis、memcached、leveldb等的实现。}
\cvitem{\hello}{熟悉推荐系统的基本算法和架构,实现过推荐系统上下游服务。}
\cvitem{\hello}{熟悉精准广告系统的基本算法和架构,了解搜索引擎的原理,参与过全文索引和广告索引系统开发。}
\cvitem{\hello}{熟练Python语言,使用Django等框架开发过web应用。用Python写过较多脚本,多数开源在github上。}
\cvitem{\hello}{熟悉Java语言,独立开发过java后台服务和分布式系统,了解Spring等框架。}
\cvitem{\hello}{熟悉JQuery/HTML/CSS和PHP,独立做过网站前端、后端设计,对网站开发和大型网站性能调优有一定经验。}
\cvitem{\hello}{熟悉基本Linux服务器运维,熟悉Linux Shell编程。}
\cvitem{\hello}{熟练使用SVN、GIT等进行代码管理,熟悉常见代码review工具、自动部署工具,熟悉VS、Eclipse、vim等开发环境和工具。开发过多个vim、sublime插件。}


\section{项目经验-工作}
\cvitem{2014.6-今 }{\textbf{ 搜狐精准广告index系统重构} \newline 维护并重构精准广告的索引系统,实现上类似于Lucence,提供对满足某个定向条件的广告候选集的搜索。C语言实现,分布式环境。}
\cvitem{2014.4-2015.5 }{\textbf{ 基于zookeeper的配置管理客户端开发} \newline 通过zookeeper管理配置项,支持在线修改和事件通知,封装了zk的java的客户端,目前已用在搜狐大数据中心多个业务。详见:\href{http://cxh.me/2015/06/16/sohu-zk-client-document/}{link}。Java实现。}
\cvitem{2014.3-2014.4 }{\textbf{ \href{http://mini.sohu.com}{搜狐微门户}视频E\&E策略index-server开发} \newline 按照策略对刚进入系统的短视频进行试探性曝光,曝光到一定次数并有点击的视频进入投放列表,参见yahoo ucb相关论文。目前已经上线并在上海地区运行。Java实现,RPC通过Thrift。}
\cvitem{2014.3-2015.4 }{\textbf{ 搜狗输入法个性化推送平台} \newline 通过聚类把用户分为多个类别,对不同类别用户编辑相关类别的热门新闻并推送。平台主要提供候选编辑,并对接搜狗的推送接口。政策原因无法做到全自动化。基于Django。}
\cvitem{2015.1-2015.3 }{\textbf{ 推荐系统各业务线监控和人工标注系统} \newline 通过thrift接口从各个业务线取数据,JOIN数据库对应新闻数据并展示以监控算法效果。提供为新闻手动打标签的界面为算法人工纠错。基于Thrift + Django。}
\cvitem{2014.11-2015.1 }{\textbf{ 手机搜狐个性频道实时UA(User Attributes)系统} \newline 根据用户最近一段时间的浏览记录给用户打标签。参见\href{http://182.92.76.83/ua/}{http://182.92.76.83/ua/}。Java实现,通过zookeeper管理server列表并实现HA。}
\cvitem{2014.2-2014.11 }{\textbf{  阿里巴巴分布式数据库Oceanbase1.0版本设计实现} \newline 参与1.0云版本事务模块的设计、分布式事务设计(两阶段提交)实现和日志同步设计(基于Paxos)。设计选主算法的测试框架。熟悉整个系统的实现原理。}
\cvitem{2013.7-2014.2 }{\textbf{ Oceanbase0.5版本重构} \newline Schema管理(异步更新Schema)、RootServer重构、内部表重构等。}


\section{项目经验-学校/创业}
\cvitem{2011.3-2011.6 }{\textbf{ 适合我旅游网} \newline 创业项目。负责了整个网站大部分的开发、运维工作,网站上线后运行良好,后因团购模式的不景气而关停。项目包含了团购、支付和酒店机票爬取、搜索(简单的垂直搜索)等。基于LAMP。代码开源在github上。}
\cvitem{2012.11-2013.3}{\textbf{ 矿山车联网监控系统} \newline 创业项目。负责软件系统。包括服务端与硬件的接口和数据存储(Python+TCP+MySQL),前台的展示和统计(LAMP)。}
\cvitem{2009.11-2010.5 }{\textbf{ 寻宝物语SNS游戏(TreasureStory)} \newline 创业项目。负责部分后台、后台管理系统开发以及数据平衡,2010年5月游戏在facebook上线之后,在Zynga该类游戏中进入Rank30,后登陆多种SNS平台。后台基于LAMP。}
\cvitem{2011.9-2012.3 }{\textbf{ 中建地产企业产品线研发数据平台} \newline 实验室项目。为中建地产􏰄供企业􏰄供企业资料、图档的分类、管理与索支持。负责整个项目的前端页面设计和后台框架搭建,实现从需求思维导图生成系统分类目录。项目基于LAMP。}
\cvitem{2010.11-2011.3 }{\textbf{ 金牌国旅企业管理系统} \newline 外包项目。为金牌国际旅行社有限公司定制企业业务流程系统以及人事、财务管理软件。领导了整个项目开发并独立设计了系统架构,项目使用.Net平台,C/S架构,数据库使用SQLServer,基于异步非阻塞Socket,目前仍在使用中。}
\cvitem{2010.7-2010.9 }{\textbf{ 某卫星地面测试系统} \newline 在航天五院负责某卫星地面测试系统开发,实现基于控件的界面编辑工具,通过串口和TCP接收数据,存储并显示。MFC框架开发。}
\cvitem{2010.4-2010.6 }{\textbf{ Admire系统MBus组件的改进和优化} \newline 本科毕设项目。主要对Admire系统(视频会议)MBus消息总线组件(跨进程、主机消息中间件)进行改进和优化,使之支持IPv6,实现新的协议,以及基于CryptoAPI实现SSL安全协议。有效提􏰄高消息路由效率和安全性。}


\section{项目经验-其他}


\end{document}
