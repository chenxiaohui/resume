\documentclass[10pt,a4paper,roman]{moderncv}

% moderncv themes
\moderncvstyle{classic}
\moderncvcolor{black}
%\renewcommand{\familydefault}{\sfdefault}
%\nopagenumbers{}
% character encoding
%\usepackage[utf8]{inputenc}
%\usepackage{CJKutf8}

% adjust the page margins
\usepackage[scale=0.85]{geometry}
\setlength{\hintscolumnwidth}{0.13\textwidth}
\setlength{\makecvtitlenamewidth}{8cm}

\usepackage{fontspec}
\usepackage{xunicode}
\usepackage{xeCJK}
\setmainfont{Times New Roman}
\setsansfont{Times New Roman}
\setmonofont{Courier New}
%\setCJKmainfont{Adobe Kaiti Std}
\setCJKmainfont[BoldFont={Kaiti SC Bold}]{Kaiti SC}
%\setCJKmainfont{Kaiti SC}
\setCJKsansfont{Kaiti SC}
\setCJKmonofont{SimHei}

\usepackage{lastpage}
\usepackage{fancyhdr}
\pagestyle{fancy}
\fancyhf{}

\usepackage{datetime2}
%\newdateformat{UKvardate}{%
%\THEDAY\ \monthname[\THEMONTH] \THEYEAR}
%\UKvardate

\usepackage{manfnt}

\newcommand{\hello}{{\tiny\textdbend}}

% to show numerical labels in the bibliography (default is to show no labels); only useful if you make citations in your resume
%\makeatletter
%\renewcommand*{\bibliographyitemlabel}{\@biblabel{\arabic{enumiv}}}
%\makeatother

% bibliography with mutiple entries
%\usepackage{multibib}
%\newcites{book,misc}{{Books},{Others}}
%----------------------------------------------------------------------------------
%            content
%----------------------------------------------------------------------------------

\AfterPreamble{\hypersetup{
    pdfstartview={XYZ null null 1.30},
    baseurl={http://}
}}

\fancyfoot[L]{\footnotesize\textit{陈晓辉的个人简历}}
\fancyfoot[C]{\footnotesize\thepage/\pageref{LastPage}}
\fancyfoot[R]{\footnotesize\textit{上次更新:\today}}

% personal data
\name{陈晓辉}{}
\title{男, 27}
\address{北京市海淀区科学院南路2号院3号楼搜狐媒体大厦 100190}{}{}
\phone[mobile]{+86~138~1179~8428}
%\phone[fixed]{+86~027~6875~5072} % optional
%\phone[fax]{+86~027~6875~4150} % optional
\email{sdqxcxh@gmail.com}
\homepage{www.cxh.me}
%\extrainfo{}
\social[github]{chenxiaohui}
\social[linkedin]{chenxiaohui}
%\photo[64pt][0.4pt]{picture}
%\quote{{\footnotesize Last updated: \today}}

\begin{document}
\makecvtitle

\section{教育经历}
\cvitem{2010.9--2013.6 }{ 清华大学软件学院,软件工程硕士, 方向:领域语义模型存储}
\cvitem{2006.9--2010.6 }{ 北京航空航大学计算机学院,计算机科学与技术本科}


\section{工作经历}
\cvitem{2014.11--今}{ 搜狐大数据中心推荐系统和精准广告团队,高级研发工程师(后台系统开发),L3.1}
\cvitem{2013.7--2014.10 }{ 支付宝北京基础数据部OceanBase团队,研发工程师(分布式数据库、分布式事务),P5}


\section{外语能力}
\cvitem{\hello}{具备良好的英语听说读写能力,熟练阅读英文文献。大学英语四级(548)、六级(549)均已通过。}


\section{获奖情况}
\cvitem{2010 }{ 北京航空航天大学优秀毕业生}
\cvitem{2010 }{ 北京航空航天大学计算机学院科技实践奖学金}
\cvitem{2009 }{ 北航第十九届“冯如杯”学生学术科技作品竞赛二等}
\cvitem{2008 }{ 北京市奥运会志愿者先进个人}


\section{个人陈述}
\cvitem{\hello}{专业兴趣浓厚,热衷于钻研各种技术,参与了很多项目,实践能力和学习能力均比较强,乐于分享,长期维护技术博客;极客,崇尚高效率快节奏的生活方式,喜欢对现有方法进行改进以提高效率。}
\cvitem{\hello}{受过规范的开发流程、做事风格和代码风格训练,有较强的团队意识和一定的组织领导能力,独立领导完成过中小型项目。扎实肯干,易于相处。}


\section{专业素质}
\cvitem{\hello}{多年C/C++经验,熟悉主要分布式算法,参与设计实现过分布式存储系统,对关系数据库理论、NoSQL有一定了解和经验。}
\cvitem{\hello}{对开源分布式存储系统有浓厚兴趣,了解redis、memcached、leveldb等的实现。}
\cvitem{\hello}{熟悉推荐系统的基本算法和架构,实现过推荐系统上下游服务。}
\cvitem{\hello}{熟悉精准广告系统的基本算法和架构,了解搜索引擎的原理,参与过全文索引和广告索引系统开发。}
\cvitem{\hello}{熟练Python语言,使用 Django等框架开发过web应用。用Python写过较多脚本,多数开源在github上。}
\cvitem{\hello}{熟悉Java语言,独立开发过java后台服务和分布式系统,了解Spring等框架。}
\cvitem{\hello}{熟悉JQuery/HTML/CSS和PHP,独立做过网站前端、后端设计,对网站开发和大型网站性能调优有一定经验。}
\cvitem{\hello}{熟悉基本Linux服务器运维,熟悉Linux Shell编程。}
\cvitem{\hello}{熟练使用SVN、GIT等进行代码管理,熟悉常见代码review工具、自动部署工具,熟悉VS、Eclipse、vim等开发环境和工具。开发过多个vim、sublime插件。}


\section{项目经验}
\cvitem{2014.4-2015.5 }{ 基于zookeeper的配置管理客户端开发 \newline  \newline 通过zookeeper管理配置项,支持在线修改和事件通知,封装了zk的java的客户端,目前已用在搜狐大数据中心多个业务。 详见: \href{http://cxh.me/2015/06/16/sohu-zk-client-document/}{link}。Java实现。}
\cvitem{2014.3-2014.4 }{ \href{http://mini.sohu.com}{搜狐微门户}视频E\&E策略index-server开发 \newline \newline 按照策略对刚进入系统的短视频进行试探性曝光,曝光到一定次数并有点击的视频进入投放列表,参见yahoo ucb相关论文。目前已经上线并在上海地区运行。Java实现,RPC通过Thrift。}
\cvitem{2014.3-2015.4 }{ 搜狗输入法个性化推送平台 \newline \newline 通过聚类把用户分为多个类别,对不同类别用户编辑相关类别的热门新闻并推送。平台主要提供候选编辑,并对接搜狗的推送接口。政策原因无法做到完全自动化。基于Django。}
\cvitem{2014.11-2015.1 }{ 手机搜狐个性频道实时UA(User Attributes)系统 \newline 根据用户最近一段时间的浏览记录给用户打标签 \newline 参见\href{http://182.92.76.83/ua/}{http://182.92.76.83/ua/}。java实现,通过zookeeper管理server列表并实现HA。}
\cvitem{2015.1-2015.3 }{ 推荐系统各业务线监控和人工标注系统 \newline  \newline 通过thrift接口从各个业务线取数据,JOIN数据库对应新闻数据并展示以监控算法效果。提供为新闻手动打标签的界面为算法人工纠错。基于thrift + django。}
\cvitem{2013.7-今}{ 阿里巴巴分布式数据库 Oceanbase \newline 0.5 版本的重构 \newline Schema 管理、RootServer 重构、内部 表重构。参与 1.0 云版本事务模块的设计、分布式事务设计 实现和日志同步设计。熟悉整个系统的实现原理。}
\cvitem{2011.3-2011.6 }{ 适合我旅游网 \newline 创业项目 \newline 负责了整个网站大部分的开发、运维工作,网站上 线后运行良好,后因团购模式的不景气而关停。项目包含了 团购、支付系统和酒店机票爬取、倒排(简单的垂直搜索) 等。基于 LAMP。代码开源在 github上。}
\cvitem{2012.11-2013.3}{ 矿山车联网监控系统 \newline 创业项目 \newline 负责软件系统。包括服务端与硬件的接口和数据 存储(python+tcp+mysql),前台的展示和统计。(LAMP)}
\cvitem{2009.11-2010.5 }{ 寻宝物语 SNS 游戏 (Treasure Story) \newline 创业项目 \newline 负责部分后台、后台管理系统开发以及数据平衡, 2010.5 月游戏在 facebook 上线之后,在 Zynga 该类游戏 中进入 Rank30,后登陆多种 SNS 平台。基于 LAMP。}
\cvitem{2011.9-2012.3 }{ 中建地产企业产品线研发数据平台 \newline 为中建地产􏰄供企业􏰄供企业资料、图档的分类、管理与检 索支持 \newline 负责整个项目的前端设计和后台框架搭建,并完成 了从􏰊述需求的思维导图直接生成系统分类目录的工作。项 目基于 LAMP。后台框架支持 Memcached 和页面模板}
\cvitem{2010.11-2011.3 }{ 金牌国旅企业管理系统 \newline 为金牌国际旅行社有限公司定制企业业务流程系统以及人 事、财务管理 \newline 领导了整个项目开发并独立设计了系统架构, 项目使用.Net 平台,基于 C/S 框架,数据库使用 SQLServer,基于异步非阻塞 Socket,目前仍在使用中。}
\cvitem{2010.7-2010.9 }{ 某卫星地面测试系统 \newline 在航天五院负责某卫星地面测试系统开发,实现基于控件的 界面编辑,通过串口和 TCP 接收数据,存储并显示,使用 MFC 框架开发 \newline }
\cvitem{2010.4-2010.6 }{ Admire 系统 MBus 组 件的改进和优化 \newline 本科毕设题目,主要对 MBus 消息总线组件(MBus 是一套 􏰄供进程/主机通信的消息总线中间件)进行改进和优化,内 容主要有 IPv6 支持,基于 CryptoAPI 的 SSL 加密以及新 的 MBus 协议的实现,有效􏰄高 MBus 组件效率和安全性 \newline }
\cvitem{2009.7-2009.9 }{ Admire 系统高清视频 采集模块设计与实现 \newline 该项目是个人生产实习项目,主要内容是对 Admire 系统 (视频会议系统)VS2 组件添加高清视频采集的支持,实现 了对 720p 和 1080p 格式视频的采集和渲染 \newline }
\cvitem{其他项目}{ C0 文法编译器、正则表达式匹配器、美国哥伦布集团网站、 Admire 系统网站的 WebService 封装、中国农村三级医疗 网远程医疗系统自动更新模块、一些 VIM 插件等 \newline }


\section{其他经历}
\cvitem{2015.7    }{ 简历生成器(基于latex+moderncv) \href{https://github.com/chenxiaohui/resume} \newline 通过文本文件生成简历tex的generator \newline  \newline 详见:http://cxh.me/2015/06/26/resume-generator-using-latex-and-moderncv/}
\cvitem{2011.4-今 }{ 清华大学软件学院 BIM 课题组 \newline 参与实验室承担的中建地产绿色产品线研发数据平台开发, 承担子课题建筑领域语义数据服务的研究(基于 NoSQL) \newline }
\cvitem{2010.7-2010.9 }{ 航天五院 503 所\newline 负责某卫星地面测试系统的设计开发,以及某卫星模拟测试软件的修改维护 \newline }
\cvitem{2008.3-2010.7 }{ 北京航空航天大学软件 开发环境国家重点实验 室 Admire 组 \newline 负责实验室承担的 973 项目子课题 Admire 系统(主要是视 频会议)的一些组件(消息总线,网站,高清采集渲染等) 的开发维护工作 \newline }


\end{document}
