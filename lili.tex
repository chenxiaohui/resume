\documentclass[10pt,a4paper,roman]{moderncv}

% moderncv themes
\moderncvstyle{classic}
\moderncvcolor{black}
%\renewcommand{\familydefault}{\sfdefault}
%\nopagenumbers{}
% character encoding
%\usepackage[utf8]{inputenc}
%\usepackage{CJKutf8}

% adjust the page margins
\usepackage[scale=0.85]{geometry}
\setlength{\hintscolumnwidth}{0.13\textwidth}
\setlength{\makecvtitlenamewidth}{8cm}

\usepackage{fontspec}
\usepackage{xunicode}
\usepackage{xeCJK}
\setmainfont{Times New Roman}
\setsansfont{Times New Roman}
\setmonofont{Courier New}
%\setCJKmainfont{Adobe Kaiti Std}
\setCJKmainfont[BoldFont={Kaiti SC Bold}]{Kaiti SC}
%\setCJKmainfont{Kaiti SC}
\setCJKsansfont{Kaiti SC}
\setCJKmonofont{SimHei}

\usepackage{lastpage}
\usepackage{fancyhdr}
\pagestyle{fancy}
\fancyhf{}

\usepackage{datetime2}
%\newdateformat{UKvardate}{%
%\THEDAY\ \monthname[\THEMONTH] \THEYEAR}
%\UKvardate

\usepackage{manfnt}

\newcommand{\hello}{{\tiny\textdbend}}

% to show numerical labels in the bibliography (default is to show no labels); only useful if you make citations in your resume
%\makeatletter
%\renewcommand*{\bibliographyitemlabel}{\@biblabel{\arabic{enumiv}}}
%\makeatother

% bibliography with mutiple entries
%\usepackage{multibib}
%\newcites{book,misc}{{Books},{Others}}
%----------------------------------------------------------------------------------
%            content
%----------------------------------------------------------------------------------

\AfterPreamble{\hypersetup{
    pdfstartview={XYZ null null 1.30},
    baseurl={http://}
}}

\fancyfoot[L]{\footnotesize\textit{郑莉莉的个人简历}}
\fancyfoot[C]{\footnotesize\thepage/\pageref{LastPage}}
\fancyfoot[R]{\footnotesize\textit{上次更新:\today}}

% personal data
\name{郑莉莉}{}
\title{女, 23}
\address{籍贯:福建莆田}{}{}
\phone[mobile]{+86~137~1797~9176}
%\phone[fixed]{+86~027~6875~5072} % optional
%\phone[fax]{+86~027~6875~4150} % optional
\email{zlily1992@163.com}

\begin{document}
\makecvtitle

\section{教育经历}
\cvitem{2010.09--2014.06 }{ 北京邮电大学, 通信工程学院,  通信工程专业本科}
\cvitem{2014.09--今 }{ 北京邮电大学  , 通信工程学院,   电子与通信工程研一}


\section{实习经历}
\cvitem{2013.11--2014.04}{           新浪无线测试部               ,  测试实习生}
\cvitem{2015.04--今 }{            搜狐大数据中心推荐系统及精准广告团队     ,     软件开发实习生}


\section{外语能力}
\cvitem{\hello}{良好的英语读写能力,通过CET-6(547)。}


\section{获奖情况}
\cvitem{2010.09-2013.12 }{    多次获得北京邮电大学校二等奖学金}
\cvitem{2012.12      }{     获得北京邮电大学校三好学生}
\cvitem{2013.05      }{      获得北京邮电大学优秀团员}



\section{专业素质}
\cvitem{\hello}{熟悉Python语言,熟悉Java基础知识,具有一定的编程能力。}
\cvitem{\hello}{熟悉基本的数据结构算法,对推荐系统基本流程有一定了解。}
\cvitem{\hello}{有MySQL数据库基础,熟悉基本SQL语句。}
\cvitem{\hello}{熟悉基本的Linux命令,了解TCP/IP协议。}
\cvitem{\hello}{有CCNA证书。}



\section{项目经验}
\cvitem[.5em]{2015.07--2015.08}{
    \textbf{ 中插广告打点后台系统实现 (python+django+celery)} \newline
    \fixitemize{
    \begin{itemize}
        \item[]  响应视频端发来的HTTP请求,异步处理提交视频打点任务,程序运行结束返回打点结果。通过队列实现并提供重试机制。
    \end{itemize}
    }
}
\cvitem[.5em]{2015.05--2015.06}{
    \textbf{ 搜狐广告数据统计及展示平台(hadoop+hive)} \newline
    \fixitemize{
    \begin{itemize}
        \item[]  编写mapreduce程序,通过hadoop统计每一个广告位的广告收益并用Python处理最后结果数据。 将用户标签数据和新闻推送日志放入hive,统计两个联合的新闻数据。前端可查询和展示。(Java+Spring+myBatis)
    \end{itemize}
    }
}
\cvitem[.5em]{2015.04--2015.07 }{
    \textbf{ 搜狐大数据中心监控平台 (python + django + mysql+bootstrap)} \newline
    \fixitemize{
    \begin{itemize}
        \item[]  参与推荐系统的的开发,主要完成的是对于各个业务线推荐数据的监控,便于分析模型的效果以及及时发现异常。平台框架爱提供良好扩展性,能快速加入新的业务线监控。
    \end{itemize}
    }
}
\cvitem[.5em]{2013.11--2014.04 }{
    \textbf{ 新浪无线测试部,自动化测试项目} \newline
    \fixitemize{
    \begin{itemize}
        \item[]  参与接口的自动化测试,主要是用的Json语句编写测试过程和测试用例。 负责微博在H5和WAP2.0页面上的功能测试。
    \end{itemize}
    }
}


\section{在校实践经历}
\cvitem[.5em]{2014.04--2014.06}{ 在SDN环境下,仿真一个web服务器集群,用POX(Python)实现服务器的负载均衡}
\cvitem[.5em]{2012.11--2012.12 }{ 基于学校提供的华为路由器和交换机建立小型的企业网,组队完成项目,我在其中完成基本拓扑的搭建}


\end{document}
